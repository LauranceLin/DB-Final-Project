\documentclass[12pt,a4paper]{article}

\usepackage{amsmath}
\usepackage{amsthm}
\usepackage{amssymb}
\usepackage{booktabs}
\usepackage{graphicx}
\usepackage{color}
\usepackage{fullpage}
\usepackage{listings}
\usepackage{color}
\usepackage{url}
\usepackage{multirow}
% 為了讓表格緊密
\usepackage{float}
% 為了首行空格
%\usepackage{indentfirst}
% % 為表格添加 footnote
% \usepackage{tablefootnote}

% for Chinese
% [Laurance] 因為 Overleaf 沒有新細明體,所以我改成 cwTeXMing
\usepackage{fontspec} % 加這個就可以設定字體
\usepackage[BoldFont, SlantFont]{xeCJK} % 讓中英文字體分開設置
\setCJKmainfont{cwTeXMing} % 設定中文為系統上的字型,而英文不去更動,使用原TeX\字型
\renewcommand{\baselinestretch}{1.3}

\parskip=5pt
\parindent=24pt
\newtheorem{lemma}{Lemma}
\newtheorem{ques}{Question}
\newtheorem{prop}{Proposition}
\newtheorem{defn}{Definition}
\newtheorem{rmk}{Remark}
\newtheorem{note}{Note}
\newtheorem{eg}{Example}
\newtheorem{aspt}{Assumption}

\definecolor{emphOrange}{RGB}{247, 80, 0}
\definecolor{stringGray}{RGB}{109, 109, 109}
\definecolor{commentGreen}{RGB}{0, 96, 0}
\definecolor{mygreen}{rgb}{0,0.6,0}
\definecolor{mygray}{rgb}{0.5,0.5,0.5}
\definecolor{mymauve}{rgb}{0.58,0,0.82}

\lstset{
  belowcaptionskip=1\baselineskip,
  breaklines=true,
  frame=L,
%  xleftmargin=\parindent,
  language = SQL,
  showstringspaces=false,
  basicstyle = \ttfamily, 
  keywordstyle = \bfseries\color{blue}, 
  emph = {symbol1, symbol2},
  emphstyle = \color{red},
  emph = {[2]symbol3, symbol4},
  emphstyle = {[2]\color{emphOrange}},
  commentstyle = \color{commentGreen}, 
  stringstyle = \color{stringGray}, 
%  backgroundcolor = \color{white}, 
%  numbers = left, % 沒有行號,複製貼上測試程式會比較方便
%  numberstyle = \normalsize, 
%	stepnumber = 1, 
%  numbersep = 10pt, 
%  title = ,
}

% "摘要", "表", "圖", "參考文獻"
\renewcommand{\abstractname}{\bf 摘要}
\renewcommand{\tablename}{表}
\renewcommand{\figurename}{圖}
\renewcommand{\refname}{\bf 參考文獻}





\begin{document}

\title{}
\author{}
\date{}
%\maketitle
%\fontsize{20}{20pt}\selectfonb08901164@ntu.edu.twt

% [Laurance] 請大家打上自己的名字!日期之後也可以再修改
\begin{center}
\textbf{\Large 資料庫管理(112-1) \\[5pt]
期末專案計劃書} \\[10pt]
B11705050 林稚翔 \\
B03611040 蕭鈺龍 \\ 
2023 年 10 月 23 日
\end{center}





\section{系統分析}

XX平台是一個可以讓使用者販售、轉讓、徵求和購買演唱會票券的平台。不論你是在尋找自己最愛的藝人的演唱會門票,或是你有閒置的演唱會票券想出售,這個平台都能滿足你的需求。我們提供安全的交易方式,以及交易回報機制來保障買賣雙方的安全。
使用者可以在此平台建立販售或徵求的貼文;買家和賣家可以對感興趣的票券貼文提出申請,而貼文的建立使用者可以在提出申請的買家或賣家中選擇想交易的對象。也會有平台管理員維護平台的交易安全性,杜絕詐騙貼文。

\subsection{系統功能}    

\subsubsection{交易流程相關設定}

在買方和賣方成功媒合後,將先由買方向平台線上支付一筆金額,包含該筆票券的價格、由賣方寄送至買方將支出的運費以及在平台交易的手續費。
在交易達成後,賣方要將票券寄出,並向平台回報物流編號,由平台向物流公司追蹤並顯示物流寄送狀態。
待買方完成取貨且確認票券品項、狀態等皆無誤後,將向平台回報交易完成,此時平台再將該筆交易金額及運費撥款至賣方。

交易也可能會失敗。如果賣方在7天內沒有出貨,或是買方在票券已送達的7天內未取貨,平台會把買方預付的款項退回,並停止賣方或買方的使用權限3個月。
而單一帳號被停權3次以上,帳號會被平台管理方刪除,如果發布的交易貼文被平台管理方視為詐騙,也將被永久停權。





\subsubsection{給 User 的功能}

在本系統中,User 可以執行以下功能:
\begin{enumerate}
\item 

註冊帳號:使用者可以提供自己的姓名、電子信箱等相關資訊註冊平台帳號,必須登入帳號後才可以在平台上進行互動。

\item 

新增交易貼文:使用者可以選擇要購買或出售票券,並填寫票券相關資訊,包含活動類型、活動名稱、活動時間、票券數量、出售價格(購買票券者毋須填寫)以及這筆交易達成的期限,當貼文發佈後,系統會給訂一個屬於該貼文的編號。

\item 

修改交易貼文:使用者可以修改票券數量、出售數量及這筆交易達成的期限,但活動類型、活動名稱與活動時間無法更動。

\item 

刪除交易貼文:使用者如果臨時不想交易票券,可以刪除由自己發出的交易貼文。

\item 

新增交易:使用者如果看到符合自己需求的交易文章,可以選擇與其交易,如果是購買票券者,需填寫收貨方式,而如果是出售票券者,需填寫出售價格。待填寫完成後,系統會給訂一個屬於該交易的編號,並更新相關貼文的票券數量,如果票券數量歸零,將不再顯示該貼文。

\item 

向平台付款:購買票券者要向平台支付一筆包含交易金額、物流費及手續費的款項。

\item 

新增物流編號:出售票券者將票券寄出後,要向平台回報其物流編號。

\item 

修改物流編號:出售票券者如果填錯物流編號,可以進行修改。

\item 

新增交易完成:出售票券者收到並確認過票券後,向平台回報交易完成。

\item 

查詢交易貼文:使用者可以查詢目前仍有票券要交易的貼文。

\item 

查詢使用者過去的交易:使用者可以查詢自己過去進行的交易。
        
\end{enumerate}
   
   

   
   
\subsubsection{給 Admin 的功能}

在本系統中,Admin 可以執行以下功能:
\begin{enumerate}
\item 

管理交易貼文:管理員可以對交易貼文進行查詢、刪除、修改的操作。

\item 

管理使用者帳號:管理員可以對使用者帳號進行查詢、刪除、修改的操作。

\item 

新增使用者帳號停權:管理員可以將使用者帳號停權。

\end{enumerate}





\subsubsection{給 Analyst 的功能}

在本系統中,Analyst 可以執行以下功能:
\begin{enumerate}
\item 

查詢使用者資訊:資料分析師可查詢所有使用者的活動紀錄,包括該使用者曾經發出的貼文、參與的交易有哪些及停權狀態。

\item 

查詢交易貼文資訊:資料分析師可查詢每一則交易貼文的詳細資訊。

\item 

查詢交易資訊:資料分析師可查詢所有交易的詳細資訊,包含付款狀態、物流狀態及交易是否完成。

\end{enumerate}





\end{document}
